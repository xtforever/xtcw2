\input wbuildmac.tex
\Class{ClipView}\Publicvars
\Table{ClipView}
XtNtext&XtCText&String &"Hello World How are you?"\cr
XtNcornerRoundPercent&XtCCornerRoundPercent&int &"20"\cr
XtNtimer&XtCTimer&int &"2000"\cr
\endTable
\Section
\Publicvar{XtNtext}
String  text = {\langle}String{\rangle}"Hello World How are you?"

\Section
\Publicvar{XtNcornerRoundPercent}
int  cornerRoundPercent = "20"

\Section
\Publicvar{XtNtimer}
int  timer = {\langle}String{\rangle}"2000"

\End\Table{Wheel}
XtNxftFont&XtCXFtFont&XftFont&"Sans-22"\cr
XtNcallback&XtCCallback&Callback&NULL \cr
XtNbg{\underline}norm&XtCBg{\underline}norm&Pixel&"lightblue"\cr
XtNbg{\underline}sel&XtCBg{\underline}sel&Pixel&"yellow"\cr
XtNbg{\underline}hi&XtCBg{\underline}hi&Pixel&"red"\cr
XtNfg{\underline}norm&XtCFg{\underline}norm&Pixel&"black"\cr
XtNfg{\underline}sel&XtCFg{\underline}sel&Pixel&"green"\cr
XtNfg{\underline}hi&XtCFg{\underline}hi&Pixel&"white"\cr
XtNuser{\underline}data&XtCUser{\underline}data&Int &0 \cr
XtNfocus{\underline}group&XtCFocus{\underline}group&String &""\cr
XtNstate&XtCState&Int &0 \cr
XtNregister{\underline}focus{\underline}group&XtCRegister{\underline}focus{\underline}group&Boolean &True \cr
\endTable
\Table{Core}
XtNx&XtCX&Position &0 \cr
XtNy&XtCY&Position &0 \cr
XtNwidth&XtCWidth&Dimension &0 \cr
XtNheight&XtCHeight&Dimension &0 \cr
borderWidth&XtCBorderWidth&Dimension &0 \cr
XtNcolormap&XtCColormap&Colormap &NULL \cr
XtNdepth&XtCDepth&Int &0 \cr
destroyCallback&XtCDestroyCallback&XTCallbackList &NULL \cr
XtNsensitive&XtCSensitive&Boolean &True \cr
XtNtm&XtCTm&XTTMRec &NULL \cr
ancestorSensitive&XtCAncestorSensitive&Boolean &False \cr
accelerators&XtCAccelerators&XTTranslations &NULL \cr
borderColor&XtCBorderColor&Pixel &0 \cr
borderPixmap&XtCBorderPixmap&Pixmap &NULL \cr
background&XtCBackground&Pixel &0 \cr
backgroundPixmap&XtCBackgroundPixmap&Pixmap &NULL \cr
mappedWhenManaged&XtCMappedWhenManaged&Boolean &True \cr
XtNscreen&XtCScreen&Screen *&NULL \cr
\endTable
\Imports
\Section
\Code
{\incl} {\langle}X11/Xatom.h{\rangle}\endCode


\Section
\Code
{\incl} {\langle}X11/Xft/Xft.h{\rangle}\endCode


\Section
\Code
{\incl} "converters-xft.h"\endCode


\Section
\Code
{\incl} {\langle}X11/Xmu/Converters.h{\rangle}\endCode


\Section
\Code
{\incl} "mls.h"\endCode


\Section
\Code
{\incl} "xutil.h"\endCode


\End\Privatevars
\Section
\Code
XftDraw * draw\endCode


\Section
\Code
Pixmap  pixmap\endCode


\Section
\Code
GC  gc{\underline}copy\endCode


\Section
\Code
int  prefered{\underline}width\endCode


\Section
\Code
int  prefered{\underline}height\endCode


\Section
\Code
int  mk{\underline}display\endCode


\Section
\Code
XtIntervalId  tmr\endCode


\End\Classvars
\Section
\Code
XtEnum  compress{\underline}exposure = TRUE \endCode


\Section
\Code
Atom  XA{\underline}UTF8{\underline}STRING = 0 \endCode


\End\Methods
\Section
\Code
initialize(Widget  request, {\dollar}, ArgList  args, Cardinal * num{\underline}args)
{\lbrace}
    TRACE(1, "ClipView");
    {\dollar}pixmap = 0;
    {\dollar}draw = 0;
    {\dollar}gc{\underline}copy =  XtGetGC({\dollar}, 0,0 );
    if( {\dollar}width == 0 ) {\dollar}width = 10;
    if( {\dollar}height == 0 ) {\dollar}height = 10;    
    calculate{\underline}size({\dollar});
    if(! {\dollar}XA{\underline}UTF8{\underline}STRING )
        {\dollar}XA{\underline}UTF8{\underline}STRING = XInternAtom(XtDisplay({\dollar}), "UTF8{\underline}STRING", True);

    {\dollar}mk{\underline}display = m{\underline}create(10,sizeof(char*));
    {\dollar}tmr=0;
{\rbrace}\endCode


\Section
\Code
realize({\dollar}, XtValueMask * mask, XSetWindowAttributes * attributes)
{\lbrace}
        TRACE(1, "ClipView");
        XtCreateWindow({\dollar}, (unsigned int) InputOutput,
                          (Visual *) CopyFromParent, *mask, attributes);
        reshape{\underline}widget({\dollar});

{\rbrace}\endCode


\Section
\Code
resize({\dollar})
{\lbrace}
    if (XtIsRealized({\dollar})) {\lbrace}
       if( {\dollar}width {\rangle} 4096 {\bar}{\bar} {\dollar}height {\rangle} 4096 ) return;
       
       reshape{\underline}widget({\dollar});
       if( {\dollar}pixmap ) XFreePixmap(XtDisplay({\dollar}),{\dollar}pixmap);
       {\dollar}pixmap=0;
       alloc{\underline}pixmap({\dollar});
    {\rbrace}
{\rbrace}\endCode


\Section
\Code
destroy({\dollar})
{\lbrace}
    TRACE(1, "ClipView");
    if( {\dollar}draw) XftDrawDestroy( {\dollar}draw );
    XtReleaseGC({\dollar},{\dollar}gc{\underline}copy);
    XFreePixmap(XtDisplay({\dollar}),{\dollar}pixmap);
    if({\dollar}mk{\underline}display) m{\underline}free{\underline}strings({\dollar}mk{\underline}display,0);
    stop{\underline}timer({\dollar});
{\rbrace}\endCode


\Section
\Code
Boolean  set{\underline}values(Widget  old, Widget  request, {\dollar}, ArgList  args, Cardinal * num{\underline}args)
{\lbrace}
    /* get{\underline}selection({\dollar}); */
      return True;
{\rbrace}\endCode


\Section
\Code
expose({\dollar}, XEvent * event, Region  region)
{\lbrace}
        
    TRACE(1, "ClipView");
    redraw({\dollar});
{\rbrace}\endCode


\Section
parameter list ($, int cmd, int val)


\Code
int  exec{\underline}command({\dollar}, int  cmd, int  val)
{\lbrace}
  return 0;
{\rbrace}\endCode


\Section
When our parent asks for this widget's preferred geometry,
simply return the geometry as wanted
Currently, the method always returns {\tt XtGeometryAlmost}. It doesn't bother
to check if the preferred geometry is equal to the current geometry (in
which case it should really return {\tt XtGeometryNo}) or if the preferred
geometry is equal to what the parent proposed (in which case a return of
{\tt XtGeometryYes} should have been given.

It seems that no harm is done by always returning {\tt XtGeometryAlmost} and
letting Xt figure out what really needs to be changed.



\Code
XtGeometryResult  query{\underline}geometry({\dollar}, XtWidgetGeometry * request, XtWidgetGeometry * reply)
{\lbrace}
    reply-{\rangle}request{\underline}mode = CWX {\bar} CWY {\bar} CWWidth {\bar} CWHeight;
    calculate{\underline}size({\dollar});
    reply-{\rangle}x=0; reply-{\rangle}y=0;
    reply-{\rangle}width =  {\dollar}prefered{\underline}width;
    reply-{\rangle}height = {\dollar}prefered{\underline}height;

    TRACE(1, "ClipView:{\percent}s: parent request is: {\percent}dx{\percent}d", XtName({\dollar}),
             request-{\rangle}width,
             request-{\rangle}width );

    TRACE(1, "ClipView:{\percent}s: our prefered size is {\percent}dx{\percent}d", XtName({\dollar}),
                  {\dollar}prefered{\underline}width,
                  {\dollar}prefered{\underline}height );

    return XtGeometryAlmost;
{\rbrace}\endCode


\End\Utilities
\Section
\Code
reshape{\underline}widget({\dollar})
{\lbrace}
        int w;
        if( {\dollar}cornerRoundPercent {\rangle}0 {\ampersand}{\ampersand} {\dollar}cornerRoundPercent {\langle} 100 ) {\lbrace}
            w = Min({\dollar}height,{\dollar}width);
            w = w * {\dollar}cornerRoundPercent / 100;
            XmuReshapeWidget( {\dollar}, XmuShapeRoundedRectangle, w, w );
        {\rbrace}

        

{\rbrace}\endCode


\Section
called by update cache if pixmap==0


\Code
alloc{\underline}pixmap({\dollar})
{\lbrace}
    TRACE(1, "ClipView");
    Display *dpy = XtDisplay({\dollar});

    TRACE(2,"ClipView:{\percent}s size:{\percent}dx{\percent}d", XtName({\dollar}),  {\dollar}width, {\dollar}height );
    {\dollar}pixmap = XCreatePixmap(dpy, XtWindow({\dollar}), {\dollar}width, {\dollar}height,
                            DefaultDepth(dpy, DefaultScreen(dpy)));
    if( {\dollar}draw ) {\lbrace}
        XftDrawChange({\dollar}draw, {\dollar}pixmap );
    {\rbrace}
    else {\lbrace}
        {\dollar}draw = XftDrawCreate(dpy, {\dollar}pixmap,
                       DefaultVisual(dpy, DefaultScreen(dpy)), None);
    {\rbrace}
{\rbrace}\endCode


\Section
outputs a simple string on the window cache pixmap


\Code
str{\underline}print({\dollar}, int  x, int  y, char * s)
{\lbrace}
   if( is{\underline}empty(s) ) return;
   XftDrawStringUtf8({\dollar}draw, {\dollar}xft{\underline}col + {\dollar}state, {\dollar}xftFont,
                             x,y+{\dollar}xftFont-{\rangle}ascent, (FcChar8*)s, strlen(s) );
{\rbrace}\endCode


\Section
redraws everything on pixmap cache


\Code
update{\underline}cache({\dollar})
{\lbrace}
    TRACE(1, "ClipView");
    if( {\dollar}pixmap == 0 ) alloc{\underline}pixmap({\dollar});

    /* clear background */
    XFillRectangle(XtDisplay({\dollar}),{\dollar}pixmap, {\dollar}gc[{\dollar}state], 0,0,{\dollar}width,{\dollar}height );

    char *s ="test widget";
    int i, x=15, y=0;
    
    TRACE(1,"ClipView:{\percent}s size:{\percent}dx{\percent}d", {\dollar}name, {\dollar}width, {\dollar}height );
    TRACE(1,"{\percent}s draw string: {\percent}s ({\percent}dx{\percent}d)",{\dollar}name, s,x,y );

    for(int i=0;i{\langle}2; i++) {\lbrace}
        y = {\dollar}xftFont-{\rangle}height * i;
        str{\underline}print( {\dollar}, x,y, v{\underline}kget({\dollar}mk{\underline}display,i) );
    {\rbrace}
{\rbrace}\endCode


\Section
\Code
redraw({\dollar})
{\lbrace}
    TRACE(1, "ClipView");
    if( !XtIsRealized({\dollar})) return;
    if( ! {\dollar}tmr ) start{\underline}timer({\dollar});
    update{\underline}cache({\dollar});
    XCopyArea(XtDisplay({\dollar}),{\dollar}pixmap, XtWindow({\dollar}), {\dollar}gc{\underline}copy,
              0,0, {\dollar}width, {\dollar}height, /* source pixmap */
              0,0 ); /* target window x,y */
{\rbrace}\endCode


\Section
\Code
calculate{\underline}size({\dollar})
{\lbrace}

        int w,h;
        xft{\underline}calc{\underline}string{\underline}size( {\dollar}, {\dollar}xftFont, {\dollar}text, {\ampersand}w, {\ampersand}h );
        
        {\dollar}prefered{\underline}width = w+4;

        {\dollar}prefered{\underline}height = (h+2) * 2;
        
{\rbrace}\endCode


\Section
\Code
int  lookup{\underline}cutboffer(Atom  code)
{\lbrace}
    int cutbuffer;
    switch ((unsigned) code) {\lbrace}
    case XA{\underline}CUT{\underline}BUFFER0:
        cutbuffer = 0;
        break;
    case XA{\underline}CUT{\underline}BUFFER1:
        cutbuffer = 1;
        break;
    case XA{\underline}CUT{\underline}BUFFER2:
        cutbuffer = 2;
        break;
    case XA{\underline}CUT{\underline}BUFFER3:
        cutbuffer = 3;
        break;
    case XA{\underline}CUT{\underline}BUFFER4:
        cutbuffer = 4;
        break;
    case XA{\underline}CUT{\underline}BUFFER5:
        cutbuffer = 5;
        break;
    case XA{\underline}CUT{\underline}BUFFER6:
        cutbuffer = 6;
        break;
    case XA{\underline}CUT{\underline}BUFFER7:
        cutbuffer = 7;
        break;
    default:
        cutbuffer = -1;
        break;
    {\rbrace}
    return cutbuffer;
{\rbrace}\endCode


\Section
callback function, called by XtGetSelectionValue()


\Code
RequestSelection({\dollar}, XtPointer  client, Atom * selection, Atom * type, XtPointer  value, unsigned  long * length, int * format)
{\lbrace}
    int row = (intptr{\underline}t) client;
    int len = *length;
    char *s = value;

    TRACE(2,"got row: {\percent}d", row );
    if( is{\underline}empty(s) {\bar}{\bar} 
        (row {\langle} 0) {\bar}{\bar} 
        (row {\rangle}= 2) )                return;    

    /* no more than 100 bytes */
    if( *length {\rangle} 100 ) s[100]=0;
    TRACE(2,"value {\percent}s",s );    
    v{\underline}kset( {\dollar}mk{\underline}display, s, row );
{\rbrace}\endCode


\Section
\Code
get{\underline}selection({\dollar})
{\lbrace}

    int row;

    row = 0;
    XtGetSelectionValue({\dollar}, XA{\underline}PRIMARY, {\dollar}XA{\underline}UTF8{\underline}STRING,
                        RequestSelection,
                        (void*)(intptr{\underline}t)row, 0);

    row = 1;
    XtGetSelectionValue({\dollar}, XA{\underline}SECONDARY, {\dollar}XA{\underline}UTF8{\underline}STRING,
                        RequestSelection,
                        (void*)(intptr{\underline}t)row, 0);
{\rbrace}\endCode


\Section
\Code
stop{\underline}timer({\dollar})
{\lbrace}
   if( {\dollar}tmr ) {\lbrace}
     XtRemoveTimeOut({\dollar}tmr);
     {\dollar}tmr=0;
   {\rbrace}
{\rbrace}\endCode


\Section
\Code
start{\underline}timer({\dollar})
{\lbrace}
    {\dollar}tmr = XtAppAddTimeOut( XtWidgetToApplicationContext({\dollar}), 1000, 
         (XtTimerCallbackProc)timer{\underline}cb, {\dollar});
{\rbrace}\endCode


\Section
\Code
timer{\underline}cb({\dollar}, XtIntervalId * id)
{\lbrace}
    get{\underline}selection({\dollar});
    {\dollar}tmr = XtAppAddTimeOut( XtWidgetToApplicationContext({\dollar}), 1000, 
          (XtTimerCallbackProc)timer{\underline}cb, {\dollar});
    redraw({\dollar});

{\rbrace}\endCode


\End\bye
