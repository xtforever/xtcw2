\input wbuildmac.tex
\Class{Wtext} Text Anzeige



\Publicvars
\Table{Wtext}
XtNtext&XtCText&String &NULL \cr
XtNfamily&XtCFamily&String &"Sans"\cr
XtNsizeSmall&XtCSizeSmall&int &8 \cr
XtNsizeMedium&XtCSizeMedium&int &14 \cr
XtNsizeLarge&XtCSizeLarge&int &20 \cr
XtNsizeHuge&XtCSizeHuge&int &38 \cr
\endTable
\Section
\Publicvar{XtNtext}
String  text = NULL 

\Section
\Publicvar{XtNfamily}
String  family = {\langle}String{\rangle}"Sans"

\Section
\Publicvar{XtNsizeSmall}
int  sizeSmall = 8 

\Section
\Publicvar{XtNsizeMedium}
int  sizeMedium = 14 

\Section
\Publicvar{XtNsizeLarge}
int  sizeLarge = 20 

\Section
\Publicvar{XtNsizeHuge}
int  sizeHuge = 38 

\End\Table{Wlabel}
XtNlabel&XtCLabel&String &NULL \cr
XtNcornerRoundPercent&XtCCornerRoundPercent&Int &0 \cr
XtNdraw{\underline}override&XtCDraw{\underline}override&XTCallbackProc &NULL \cr
XtNheightIncreasePercent&XtCHeightIncreasePercent&Int &0 \cr
XtNleftOffsetPercent&XtCLeftOffsetPercent&Int &5 \cr
XtNupdate&XtCUpdate&Int &0 \cr
\endTable
\Table{Wheel}
XtNxftFont&XtCXFtFont&XftFont&"Sans-22"\cr
XtNcallback&XtCCallback&Callback&NULL \cr
XtNbg{\underline}norm&XtCBg{\underline}norm&Pixel&"lightblue"\cr
XtNbg{\underline}sel&XtCBg{\underline}sel&Pixel&"yellow"\cr
XtNbg{\underline}hi&XtCBg{\underline}hi&Pixel&"red"\cr
XtNfg{\underline}norm&XtCFg{\underline}norm&Pixel&"black"\cr
XtNfg{\underline}sel&XtCFg{\underline}sel&Pixel&"green"\cr
XtNfg{\underline}hi&XtCFg{\underline}hi&Pixel&"white"\cr
XtNuser{\underline}data&XtCUser{\underline}data&Int &0 \cr
XtNfocus{\underline}group&XtCFocus{\underline}group&String &""\cr
XtNstate&XtCState&Int &0 \cr
XtNregister{\underline}focus{\underline}group&XtCRegister{\underline}focus{\underline}group&Boolean &True \cr
\endTable
\Table{Core}
XtNx&XtCX&Position &0 \cr
XtNy&XtCY&Position &0 \cr
XtNwidth&XtCWidth&Dimension &0 \cr
XtNheight&XtCHeight&Dimension &0 \cr
borderWidth&XtCBorderWidth&Dimension &0 \cr
XtNcolormap&XtCColormap&Colormap &NULL \cr
XtNdepth&XtCDepth&Int &0 \cr
destroyCallback&XtCDestroyCallback&XTCallbackList &NULL \cr
XtNsensitive&XtCSensitive&Boolean &True \cr
XtNtm&XtCTm&XTTMRec &NULL \cr
ancestorSensitive&XtCAncestorSensitive&Boolean &False \cr
accelerators&XtCAccelerators&XTTranslations &NULL \cr
borderColor&XtCBorderColor&Pixel &0 \cr
borderPixmap&XtCBorderPixmap&Pixmap &NULL \cr
background&XtCBackground&Pixel &0 \cr
backgroundPixmap&XtCBackgroundPixmap&Pixmap &NULL \cr
mappedWhenManaged&XtCMappedWhenManaged&Boolean &True \cr
XtNscreen&XtCScreen&Screen *&NULL \cr
\endTable
\Exports
\Section
\Code
{\incl} "util.h"\endCode


\End\Imports
\Section
\Code
{\incl} {\langle}X11/Xft/Xft.h{\rangle}\endCode


\Section
\Code
{\incl} "converters-xft.h"\endCode


\Section
\Code
{\incl} {\langle}X11/Xmu/Converters.h{\rangle}\endCode


\Section
\Code
{\incl} "mls.h"\endCode


\Section
\Code
{\incl} "util.h"\endCode


\Section
\Code
{\incl} "xutil.h"\endCode


\End\Privatevars
\Section
\Code
Boolean  relayout\endCode


\Section
\Code
int  update\endCode


\Section
\Code
tseg{\underline}text{\underline}t  tseg\endCode


\Section
\Code
int  total{\underline}height\endCode


\Section
\Code
int  seg{\underline}w\endCode


\Section
\Code
int  seg{\underline}h\endCode


\End\Methods
\Section
\Code
initialize(Widget  request, {\dollar}, ArgList  args, Cardinal * num{\underline}args)
{\lbrace}
    {\dollar}draw{\underline}override = sensor{\underline}draw;       
    {\dollar}relayout = True;      
    {\dollar}update = -1; 
    if( {\dollar}text == NULL ) {\dollar}text = {\dollar}name;
    init{\underline}text({\dollar});
    set{\underline}text({\dollar}, {\dollar}text);
    /* measure{\underline}segments */
    {\dollar}width = {\dollar}seg{\underline}w; {\dollar}height = {\dollar}seg{\underline}h; 
    TRACE(2,"Wtext: {\percent}s Height: {\percent}d", {\dollar}name, {\dollar}height );
{\rbrace}\endCode


\Section
\Code
destroy({\dollar})
{\lbrace}
    text{\underline}attribute{\underline}t *a; int i;
    m{\underline}foreach( {\dollar}tseg.m{\underline}attr, i, a )
        XftFontClose( XtDisplay({\dollar}), a-{\rangle}font );
    tseg{\underline}free( {\ampersand} {\dollar}tseg );
{\rbrace}\endCode


\Section
\Code
resize({\dollar})
{\lbrace}
    
    {\dollar}relayout = True; 
{\rbrace}\endCode


\Section
\Code
Boolean  set{\underline}values(Widget  old, Widget  request, {\dollar}, ArgList  args, Cardinal * num{\underline}args)
{\lbrace}  
    return True;
{\rbrace}\endCode


\End\Utilities
\Section
erzeuge eine liste von text segmenten, kein segment ist größer als 




\Section
als das angegebene rectangle breit ist. die höhe kann variieren.


\Code
int  word{\underline}wrap(int  seg{\underline}list, int  txt, XRectangle * r)
{\lbrace}
    return 0;
{\rbrace}\endCode


\Section
\Code
int  get{\underline}height({\dollar}, int  attribute)
{\lbrace}
    text{\underline}attribute{\underline}t *attr = mls( {\dollar}tseg.m{\underline}attr, attribute ); 
    return attr-{\rangle}font-{\rangle}height;
{\rbrace}\endCode


\Section
\Code
int  get{\underline}width({\dollar}, int  attribute, int  ms, int  start, int  end)
{\lbrace}
    char *s = mls(ms,start);
    char *p = mls(ms,end);
    XGlyphInfo      extents;
    text{\underline}attribute{\underline}t *attr = mls( {\dollar}tseg.m{\underline}attr, attribute ); 
    
    XftTextExtentsUtf8(XtDisplay({\dollar}), attr-{\rangle}font,  (FcChar8*)  s,
                       (int)(p-s)+1, {\ampersand}extents );
    return extents.xOff;
{\rbrace}\endCode


\Section
\Code
measure{\underline}segments({\dollar})
{\lbrace}
    text{\underline}segment{\underline}t *seg;
    int i;
    int w=0, h=0;
    m{\underline}foreach({\dollar}tseg.m{\underline}seg,i,seg) {\lbrace}
        w+=seg-{\rangle}r.width = get{\underline}width({\dollar}, seg-{\rangle}attribute, 
                                 {\dollar}tseg.text, seg-{\rangle}start, seg-{\rangle}end );
        h+=seg-{\rangle}r.height = get{\underline}height({\dollar}, seg-{\rangle}attribute );
    {\rbrace}
    {\dollar}seg{\underline}w = w; {\dollar}seg{\underline}h = h;
{\rbrace}\endCode


\Section
\Code
void  write{\underline}int(int  m, int  p, int  val)
{\lbrace}
    int c = val, i=0;
    
    while( c {\rangle} 9 ) {\lbrace}
        c /= 10;
        i++;
    {\rbrace}
    
    m{\underline}setlen( m, p + i + 2 );
    CHAR(m, p+i+1 ) = 0;
    while( i{\rangle}=0 ) {\lbrace}
        CHAR( m, p+i ) = '0' + (val {\percent} 10);
        val /=  10;
        i--;
    {\rbrace}
{\rbrace}\endCode


\Section
\Code
init{\underline}text({\dollar})
{\lbrace}
    tseg{\underline}init({\ampersand} {\dollar}tseg );

    int len, m = m{\underline}create(50,1);
    m{\underline}write( m,0, {\dollar}family, strlen({\dollar}family));
    m{\underline}putc(  m, '-' ); len = m{\underline}len(m);

    write{\underline}int( m, len, {\dollar}sizeSmall );
    add{\underline}attrib({\dollar}, 's', mls(m,0) );

    write{\underline}int( m, len, {\dollar}sizeMedium );
    add{\underline}attrib({\dollar}, 'm', mls(m,0) );

    write{\underline}int( m, len, {\dollar}sizeLarge );
    add{\underline}attrib({\dollar}, 'l', mls(m,0) );

    write{\underline}int( m, len, {\dollar}sizeHuge );
    add{\underline}attrib({\dollar}, 'h', mls(m,0) );

    m{\underline}free(m);
{\rbrace}\endCode


\Section
\Code
add{\underline}attrib({\dollar}, char  name, char * fnt)
{\lbrace}
    Display *dpy = XtDisplay({\dollar});
    XGlyphInfo extents;
    text{\underline}attribute{\underline}t attr;
    memset( {\ampersand}attr,0,sizeof(attr));
    attr.font = xft{\underline}fontopen(dpy, DefaultScreen(dpy), fnt, False, 0 );
    memcpy( {\ampersand} attr.color, {\dollar}xft{\underline}col, sizeof( attr.color ));
    XftTextExtentsUtf8(dpy, attr.font, (FcChar8*)" ",1, {\ampersand}extents);
    attr.space{\underline}width = extents.xOff;
    attr.name = name;
    m{\underline}put( {\dollar}tseg.m{\underline}attr, {\ampersand}attr );
{\rbrace}\endCode


\Section
\Code
set{\underline}text({\dollar}, char * s)
{\lbrace}
    tseg{\underline}set{\underline}text({\ampersand} {\dollar}tseg,s );
    measure{\underline}segments({\dollar});
{\rbrace}\endCode


\Section
\Code
layout{\underline}segments({\dollar})
{\lbrace}
    int m{\underline}seg = {\dollar}tseg.m{\underline}seg;
    text{\underline}segment{\underline}t *seg;
    int i;
    int x,y;
    int max{\underline}h;
    int line{\underline}break = 0;
    {\dollar}total{\underline}height = 0;

    x=0; y=0; max{\underline}h=0;
    m{\underline}foreach(m{\underline}seg,i,seg) {\lbrace}
        if( x+seg-{\rangle}r.width {\rangle} {\dollar}width {\bar}{\bar} line{\underline}break ) {\lbrace}
            y+=max{\underline}h; x=0; max{\underline}h=0; line{\underline}break = 0;
        {\rbrace}
        max{\underline}h = Max(seg-{\rangle}r.height, max{\underline}h );
        seg-{\rangle}r.x = x;
        seg-{\rangle}r.y = y; {\dollar}total{\underline}height = y + max{\underline}h;
        x += seg-{\rangle}r.width + attribute{\underline}space( {\dollar}, seg-{\rangle}attribute );
        if( seg-{\rangle}format {\ampersand} T{\underline}HARD{\underline}BREAK) line{\underline}break=1;
    {\rbrace}       
{\rbrace}\endCode


\Section
\Code
int  attribute{\underline}space({\dollar}, int  attr{\underline}id)
{\lbrace}
    text{\underline}attribute{\underline}t *attr = mls({\dollar}tseg.m{\underline}attr, attr{\underline}id);
    return attr-{\rangle}space{\underline}width;
{\rbrace}\endCode


\Section
\Code
relayout({\dollar})
{\lbrace}
        layout{\underline}segments({\dollar});
{\rbrace}\endCode


\Section
\Code
print{\underline}str{\underline}west({\dollar}, XftFont * fnt, XRectangle * r, char * s)
{\lbrace}
    int x,y;
    if( !s ) return; /* disabled */
    XFillRectangle( XtDisplay({\dollar}), XtWindow({\dollar}), {\dollar}gc[0], 
                    r-{\rangle}x, r-{\rangle}y, r-{\rangle}width, r-{\rangle}height );
    /* XftDrawSetClipRectangles( {\dollar}draw,0,0,r, 1); */
    /*XftTextExtentsUtf8(XtDisplay({\dollar}), {\dollar}fnt, (FcChar8*)s,
                       strlen(s), {\ampersand}extents);
    w = extents.xOff;
    */

    x = 1;
    y = (r-{\rangle}height - fnt-{\rangle}height) / 2 + fnt-{\rangle}ascent;
    XftDrawStringUtf8({\dollar}draw, {\dollar}xft{\underline}col, fnt, 
                      x+r-{\rangle}x,y+r-{\rangle}y, (FcChar8*)s, strlen(s) );
{\rbrace}\endCode


\Section
\Code
print{\underline}str{\underline}center({\dollar}, XftFont * fnt, XRectangle * r, char * s)
{\lbrace}
    XGlyphInfo extents;
    int w,x,y;

    if( !s ) return; /* disabled */

    XFillRectangle( XtDisplay({\dollar}), XtWindow({\dollar}), {\dollar}gc[0], 
                    r-{\rangle}x, r-{\rangle}y, r-{\rangle}width, r-{\rangle}height );
   /* XftDrawSetClipRectangles( {\dollar}draw,0,0,r, 1); */
   XftTextExtentsUtf8(XtDisplay({\dollar}), fnt, (FcChar8*)s,
                       strlen(s), {\ampersand}extents);
    w = extents.xOff;
   
    x = (r-{\rangle}width - w) / 2;
    y = (r-{\rangle}height - fnt-{\rangle}height) / 2 + fnt-{\rangle}ascent;
    XftDrawStringUtf8({\dollar}draw, {\dollar}xft{\underline}col, fnt, 
                      x+r-{\rangle}x,y+r-{\rangle}y, (FcChar8*)s, strlen(s) );
{\rbrace}\endCode


\Section
\Code
print{\underline}int({\dollar}, XftFont * fnt, XRectangle * r, int  val)
{\lbrace}
    char buf[100];
    snprintf( buf,100,"{\percent}d", val );
    int x,y;
    char *s;

    if( val {\langle} 0 ) return; /* disabled */

    XFillRectangle( XtDisplay({\dollar}), XtWindow({\dollar}), {\dollar}gc[0], 
                    r-{\rangle}x, r-{\rangle}y, r-{\rangle}width, r-{\rangle}height );
    /* XftDrawSetClipRectangles( {\dollar}draw,0,0,r, 1); */
    x = 1;
    y = (r-{\rangle}height - fnt-{\rangle}height) / 2 + fnt-{\rangle}ascent;
    s = buf;
    XftDrawStringUtf8({\dollar}draw, {\dollar}xft{\underline}col, fnt, 
                      x+r-{\rangle}x,y+r-{\rangle}y, (FcChar8*)s, strlen(s) );
{\rbrace}\endCode


\Section
\Code
sensor{\underline}draw({\dollar}, void * a, void * b)
{\lbrace}
        
        XFillRectangle( XtDisplay({\dollar}), XtWindow({\dollar}), {\dollar}gc[ {\dollar}state ], 
                    0,0, {\dollar}width, {\dollar}height );

        if( {\dollar}relayout ) {\lbrace}
            relayout({\dollar});
            {\dollar}relayout = False;
        {\rbrace}       

        {\dollar}update = -1;
        redraw({\dollar});
{\rbrace}\endCode


\Section
\Code
print{\underline}segment({\dollar}, text{\underline}segment{\underline}t * s)
{\lbrace}
    FcChar8* str;
    int len;
    int x,y;
    text{\underline}attribute{\underline}t *attr = mls({\dollar}tseg.m{\underline}attr, s-{\rangle}attribute);

    x = s-{\rangle}r.x;
    y = s-{\rangle}r.y + attr-{\rangle}font-{\rangle}ascent;

    str = mls( {\dollar}tseg.text, s-{\rangle}start);
    len = s-{\rangle}end - s-{\rangle}start + 1;
    
    XftDrawStringUtf8({\dollar}draw, {\ampersand} attr-{\rangle}color, attr-{\rangle}font, 
                      x,y,str,len );
    
{\rbrace}\endCode


\Section
\Code
redraw({\dollar})
{\lbrace}
    text{\underline}segment{\underline}t *seg;
    int i;

    m{\underline}foreach( {\dollar}tseg.m{\underline}seg, i, seg )
        {\lbrace}
            print{\underline}segment( {\dollar}, seg );
        {\rbrace}

    {\dollar}update = 0;
{\rbrace}\endCode


\End\bye
