\input wbuildmac.tex
\Class{Wedit}Editor Widget







\Publicvars
\Table{Wedit}
XtNfontName&XtCFontName&String &"Source Code Pro-22"\cr
XtNforeground&XtCForeground&Pixel &XtDefaultForeground \cr
XtNcursorColor&XtCCursorColor&Pixel &XtDefaultForeground \cr
XtNlineWidth&XtCLineWidth&int &1 \cr
XtNcornerRoundPercent&XtCCornerRoundPercent&int &20 \cr
XtNcallback&XtCCallback&Callback&NULL \cr
XtNlabel&XtCLabel&String &NULL \cr
\endTable
\Section
\Publicvar{XtNfontName}
String  fontName = {\langle}String{\rangle}"Source Code Pro-22"

\Section
\Publicvar{XtNforeground}
Pixel  foreground = XtDefaultForeground 

\Section
\Publicvar{XtNcursorColor}
Pixel  cursorColor = XtDefaultForeground 

\Section
\Publicvar{XtNlineWidth}
int  lineWidth = 1 

\Section
\Publicvar{XtNcornerRoundPercent}
int  cornerRoundPercent = 20 

\Section
\Publicvar{XtNcallback}
{\langle}Callback{\rangle} XtCallbackList  callback = NULL 

\Section
\Publicvar{XtNlabel}
String  label = NULL 

\End\Table{Core}
XtNx&XtCX&Position &0 \cr
XtNy&XtCY&Position &0 \cr
XtNwidth&XtCWidth&Dimension &0 \cr
XtNheight&XtCHeight&Dimension &0 \cr
borderWidth&XtCBorderWidth&Dimension &0 \cr
XtNcolormap&XtCColormap&Colormap &NULL \cr
XtNdepth&XtCDepth&Int &0 \cr
destroyCallback&XtCDestroyCallback&XTCallbackList &NULL \cr
XtNsensitive&XtCSensitive&Boolean &True \cr
XtNtm&XtCTm&XTTMRec &NULL \cr
ancestorSensitive&XtCAncestorSensitive&Boolean &False \cr
accelerators&XtCAccelerators&XTTranslations &NULL \cr
borderColor&XtCBorderColor&Pixel &0 \cr
borderPixmap&XtCBorderPixmap&Pixmap &NULL \cr
background&XtCBackground&Pixel &0 \cr
backgroundPixmap&XtCBackgroundPixmap&Pixmap &NULL \cr
mappedWhenManaged&XtCMappedWhenManaged&Boolean &True \cr
XtNscreen&XtCScreen&Screen *&NULL \cr
\endTable
\Translations
\Section
\Code
{\langle}Key{\rangle}Right: forward{\underline}char() \endCode


\Section
\Code
{\langle}Key{\rangle}Left: backward{\underline}char() \endCode


\Section
\Code
{\langle}Key{\rangle}Delete: remove{\underline}char() \endCode


\Section
\Code
{\langle}Key{\rangle}BackSpace: backward{\underline}char() remove{\underline}char() \endCode


\Section
\Code
{\langle}Ctrl{\rangle}x: info() \endCode


\Section
\Code
{\langle}Ctrl{\rangle}v: insert{\underline}selection() \endCode


\Section
\Code
{\langle}Btn1Down{\rangle}: set{\underline}cursor() \endCode


\Section
\Code
{\langle}Key{\rangle}Return: notify() \endCode


\Section
\Code
{\langle}Key{\rangle}: insert{\underline}char() \endCode


\End\Actions
\Section
\Action{info}\Code
void info({\dollar}, XEvent* event, String* params, Cardinal* num{\underline}params)
{\lbrace}
    dump({\dollar});
{\rbrace}\endCode


\Section
\Action{backward{\underline}char}\Code
void backward{\underline}char({\dollar}, XEvent* event, String* params, Cardinal* num{\underline}params)
{\lbrace}
    cursor{\underline}left({\dollar});
    if( {\dollar}dirty ) paint({\dollar}); else draw{\underline}cursor({\dollar});
{\rbrace}\endCode


\Section
\Action{forward{\underline}char}\Code
void forward{\underline}char({\dollar}, XEvent* event, String* params, Cardinal* num{\underline}params)
{\lbrace}
    cursor{\underline}right({\dollar});
    if( {\dollar}dirty ) paint({\dollar}); else draw{\underline}cursor({\dollar});
{\rbrace}\endCode


\Section
\Action{insert{\underline}char}\Code
void insert{\underline}char({\dollar}, XEvent* event, String* params, Cardinal* num{\underline}params)
{\lbrace}
    int len;
    unsigned char buf[32];
    KeySym key{\underline}sim;

    len = {\underline}XawLookupString({\dollar},(XKeyEvent *) event, (char*)buf,sizeof buf, {\ampersand}key{\underline}sim );
    if (len {\langle}= 0) return;

    insert{\underline}char{\underline}at{\underline}cursor({\dollar}, (char*)buf, len );
    cursor{\underline}right({\dollar});
    paint({\dollar});
{\rbrace}\endCode


\Section
\Action{remove{\underline}char}\Code
void remove{\underline}char({\dollar}, XEvent* event, String* params, Cardinal* num{\underline}params)
{\lbrace}
    remove{\underline}char{\underline}at{\underline}cursor({\dollar});
    paint({\dollar});
{\rbrace}\endCode


\Section
\Action{set{\underline}cursor}\Code
void set{\underline}cursor({\dollar}, XEvent* event, String* params, Cardinal* num{\underline}params)
{\lbrace}
    int x = event-{\rangle}xbutton.x;
    int y = event-{\rangle}xbutton.y;
    x -= {\dollar}pix.r.x;
    y -= {\dollar}pix.r.y;
    int p = {\dollar}left{\underline}char{\underline}offset;
    int pix = 0;
    do {\lbrace}
        pix += INT({\dollar}glyph{\underline}xoff{\underline}m, p);
        if( x {\langle} pix ) break;
        p++;
    {\rbrace} while( p {\langle} (m{\underline}len({\dollar}glyph{\underline}xoff{\underline}m)-1) );
    {\dollar}crsr{\underline}pos = p;
    draw{\underline}cursor({\dollar});
{\rbrace}\endCode


\Section
\Action{insert{\underline}selection}insert_selection(Widget aw, XEvent * event, String * params, Cardinal * num_params)



\Code
void insert{\underline}selection({\dollar}, XEvent* event, String* params, Cardinal* num{\underline}params)
{\lbrace}
    XtGetSelectionValue({\dollar}, XA{\underline}PRIMARY, {\dollar}XA{\underline}UTF8{\underline}STRING,
                        RequestSelection,
                        NULL, 0 );
{\rbrace}\endCode


\Section
\Action{notify}\Code
void notify({\dollar}, XEvent* event, String* params, Cardinal* num{\underline}params)
{\lbrace}
  XtCallCallbackList( {\dollar}, {\dollar}callback, event );
{\rbrace}\endCode


\End\Imports
\Section
\Code
{\incl} {\langle}assert.h{\rangle}\endCode


\Section
\Code
{\incl} {\langle}X11/Intrinsic.h{\rangle}\endCode


\Section
\Code
{\incl} {\langle}X11/Xatom.h{\rangle}\endCode


\Section
\Code
{\incl} {\langle}X11/Xmu/Converters.h{\rangle}\endCode


\Section
\Code
{\incl} {\langle}X11/Xft/Xft.h{\rangle}\endCode


\Section
\Code
{\incl} {\langle}X11/Xaw/XawImP.h{\rangle}\endCode


\Section
\Code
{\incl} "converters.h"\endCode


\Section
\Code
{\incl} "xutil.h"\endCode


\Section
\Code
{\incl} "mls.h"\endCode


\Section
\Code
{\incl} "font.h"\endCode


\End\Privatevars
\Section
\Code
 BG\endCode


\Section
\Code
 FG\endCode


\Section
\Code
 CRSR\endCode


\Section
\Code
struct {\lbrace} Pixmap data; XRectangle r; {\rbrace} pix{\underline}info\endCode


\Section
\Code
XftFont * xftFont\endCode


\Section
\Code
pix{\underline}info  pix\endCode


\Section
\Code
int  str{\underline}m\endCode


\Section
\Code
int  label{\underline}m\endCode


\Section
\Code
int  glyph{\underline}m\endCode


\Section
\Code
int  glyph{\underline}xoff{\underline}m\endCode


\Section
\Code
XftDraw * draw\endCode


\Section
\Code
GC  gc[3]\endCode


\Section
\Code
XftColor  xcol[3]\endCode


\Section
\Code
XRectangle  crsr{\underline}rect\endCode


\Section
\Code
XRectangle  txt\endCode


\Section
\Code
char  crsr{\underline}on\endCode


\Section
\Code
int  crsr{\underline}pos\endCode


\Section
\Code
int  crsr{\underline}width\endCode


\Section
\Code
int  left{\underline}char{\underline}offset\endCode


\Section
\Code
int  left{\underline}pixel{\underline}offset\endCode


\Section
\Code
int  dirty\endCode


\Section
\Code
int  dirty{\underline}str\endCode


\End\Classvars
\Section
\Code
Atom  XA{\underline}UTF8{\underline}STRING = 0 \endCode


\End\Methods
\Section
\Code
initialize(Widget  request, {\dollar}, ArgList  args, Cardinal * num{\underline}args)
{\lbrace}
  Display *dpy = XtDisplay({\dollar});
 /*
     {\dollar}xftFont = 0;
  */
 {\dollar}draw = 0;
 // {\dollar}xftFont = XftFontOpenName( dpy, DefaultScreen(dpy), {\dollar}fontName );
  {\dollar}xftFont = font{\underline}load(dpy, DefaultScreen(dpy), {\dollar}fontName );
  int w,h;
  alloc{\underline}colors({\dollar});
  if(! {\dollar}label ) {\dollar}label = {\dollar}name;
  {\dollar}glyph{\underline}m = m{\underline}create( 50, sizeof(XftGlyphFontSpec) );
  {\dollar}glyph{\underline}xoff{\underline}m = m{\underline}create( 50, sizeof(int) );
  {\dollar}str{\underline}m  = m{\underline}create( 50, 1 );
  {\dollar}label{\underline}m = m{\underline}create( 50, 1 );
  char *s = {\dollar}label; if( is{\underline}empty(s) ) s={\dollar}name;
  copy{\underline}str( {\dollar}, s );

  calculate{\underline}prefered{\underline}size({\dollar},{\ampersand}w,{\ampersand}h);
  if( {\dollar}width {\langle} w ) {\dollar}width = w;
  if( {\dollar}height {\langle} h) {\dollar}height = h;
  if(! {\dollar}XA{\underline}UTF8{\underline}STRING )
      {\dollar}XA{\underline}UTF8{\underline}STRING = XInternAtom(XtDisplay({\dollar}), "UTF8{\underline}STRING", True);
{\rbrace}\endCode


\Section
\Code
destroy({\dollar})
{\lbrace}
        free{\underline}colors({\dollar});
        XftDrawDestroy( {\dollar}draw );
        m{\underline}free({\dollar}glyph{\underline}m);
        m{\underline}free({\dollar}str{\underline}m);
        m{\underline}free({\dollar}glyph{\underline}xoff{\underline}m);
        m{\underline}free({\dollar}label{\underline}m);
{\rbrace}\endCode


\Section
\Code
Boolean  set{\underline}values(Widget  old, Widget  request, {\dollar}, ArgList  args, Cardinal * num{\underline}args)
{\lbrace}
    int i;
    for(i=0;i{\langle}*num{\underline}args;i++) {\lbrace}
        if( strcmp(args[i].name, "label" ) ==0 )
            {\lbrace}
                copy{\underline}str({\dollar}, {\dollar}label );
            {\rbrace}
    {\rbrace}
    reinit({\dollar});
    return True; /* call expose() to update widget */
{\rbrace}\endCode


\Section
\Code
get{\underline}values{\underline}hook({\dollar}, ArgList  args, Cardinal * num{\underline}args)
{\lbrace}
    int i;
    for(i=0;i{\langle}*num{\underline}args;i++) {\lbrace}
        if( strcmp(args[i].name, "label" ) ==0 )
            {\lbrace}
                int l = m{\underline}len({\dollar}str{\underline}m);
                m{\underline}write({\dollar}label{\underline}m, 0, m{\underline}buf({\dollar}str{\underline}m), l );
                CHAR({\dollar}label{\underline}m, l-1)=0;
                *((char**)args[i].value) = m{\underline}buf({\dollar}label{\underline}m);
                return;
            {\rbrace}
    {\rbrace}
{\rbrace}\endCode


\Section
\Code
expose({\dollar}, XEvent * event, Region  region)
{\lbrace}
  Display *dpy = XtDisplay({\dollar});
  XClearWindow( dpy, XtWindow({\dollar}) );
  {\dollar}crsr{\underline}on = 0; /* cursor not visible */
  paint({\dollar});
{\rbrace}\endCode


\Section
\Code
realize({\dollar}, XtValueMask * mask, XSetWindowAttributes * attributes)
{\lbrace}
        XtCreateWindow({\dollar}, (unsigned int) InputOutput,
                          (Visual *) CopyFromParent, *mask, attributes);
        resize{\underline}widget({\dollar});
{\rbrace}\endCode


\Section
\Code
resize({\dollar})
{\lbrace}
    if (XtIsRealized({\dollar})) resize{\underline}widget({\dollar});
{\rbrace}\endCode


\End\Utilities
\Section
\Code
resize{\underline}widget({\dollar})
{\lbrace}
        /*
        int w;
        if( {\dollar}cornerRoundPercent {\rangle}0 {\ampersand}{\ampersand} {\dollar}cornerRoundPercent {\langle} 100 ) {\lbrace}
            w = Min({\dollar}height,{\dollar}width);
            w = w * {\dollar}cornerRoundPercent / 100;
            XmuReshapeWidget( {\dollar}, XmuShapeRoundedRectangle, w, w );
        {\rbrace}
        */

    if( {\dollar}pix.data ) {\lbrace} XFreePixmap( XtDisplay({\dollar}), {\dollar}pix.data ); {\rbrace}
    {\dollar}pix.data = XtCreatePixmap({\dollar},{\dollar}width,{\dollar}txt.height);
    {\dollar}pix.r.width = {\dollar}width; {\dollar}pix.r.height = {\dollar}txt.height;

    {\dollar}pix.r.x = 0;  {\dollar}pix.r.y = ({\dollar}height - {\dollar}txt.height) / 2;
    if( ! {\dollar}draw ) {\lbrace}
        {\dollar}draw = XtXftDrawCreate({\dollar},{\dollar}pix.data);
    {\rbrace}
    else {\lbrace}
        XftDrawChange({\dollar}draw,{\dollar}pix.data);
    {\rbrace}

    {\dollar}left{\underline}char{\underline}offset = 0;
    {\dollar}crsr{\underline}pos = 0;
{\rbrace}\endCode


\Section
\Code
clear{\underline}pixmap({\dollar})
{\lbrace}
   XFillRectangle(XtDisplay({\dollar}),{\dollar}pix.data,
                       {\dollar}gc[BG], 0,0, {\dollar}pix.r.width, {\dollar}pix.r.height );
{\rbrace}\endCode


\Section
\Code
draw{\underline}text{\underline}on{\underline}pixmap({\dollar})
{\lbrace}
    /* str{\underline}to{\underline}glyphlist({\dollar}); */
    draw{\underline}glyphs({\dollar});
{\rbrace}\endCode


\Section
\Code
draw{\underline}pixmap{\underline}on{\underline}window({\dollar})
{\lbrace}
        XtCopyArea({\dollar}, {\dollar}pix.data, XtWindow({\dollar}),
                   0,0, {\dollar}pix.r.width, {\dollar}pix.r.height, /* src */
                   {\dollar}pix.r.x, {\dollar}pix.r.y );             /* dest */
{\rbrace}\endCode


\Section
\Code
XftGlyphFontSpec * get{\underline}glyph{\underline}spec({\dollar}, int  p)
{\lbrace}
        return mls({\dollar}glyph{\underline}m, p);
{\rbrace}\endCode


\Section
calculate cursor rectangle and draw cursor on window


\Code
draw{\underline}cursor({\dollar})
{\lbrace}
        XRectangle r;
        if( {\dollar}crsr{\underline}pos {\rangle}= m{\underline}len({\dollar}glyph{\underline}m) )
            {\dollar}crsr{\underline}pos=m{\underline}len({\dollar}glyph{\underline}m)-1;

        r.width = INT({\dollar}glyph{\underline}xoff{\underline}m,{\dollar}crsr{\underline}pos);
        r.x = screen{\underline}px({\dollar}, {\dollar}crsr{\underline}pos) + {\dollar}pix.r.x;
        r.y = {\dollar}pix.r.y;
        r.height = {\dollar}xftFont-{\rangle}height;
        cursor{\underline}xdraw( {\dollar}, {\ampersand}r );
{\rbrace}\endCode


\Section
\Code
dump({\dollar})
{\lbrace}
    int i; for(i=0;i{\langle} m{\underline}len({\dollar}str{\underline}m)-1; i++)
               printf("{\percent}x ", (unsigned char) CHAR({\dollar}str{\underline}m, i) );
    putc(10,stdout);
    fwrite( m{\underline}buf({\dollar}str{\underline}m), m{\underline}len({\dollar}str{\underline}m), 1, stdout );
    printf("Left:{\percent}d Crsr:{\percent}d{\backslash}n", {\dollar}left{\underline}char{\underline}offset, {\dollar}crsr{\underline}pos );
{\rbrace}\endCode


\Section
clear the pixmap, draw text on pixmap, draw pixmap on window


\Code
paint({\dollar})
{\lbrace}
        clear{\underline}pixmap({\dollar});
        draw{\underline}text{\underline}on{\underline}pixmap({\dollar});
        draw{\underline}pixmap{\underline}on{\underline}window({\dollar}); {\dollar}crsr{\underline}on = 0;
        draw{\underline}cursor({\dollar});
        {\dollar}dirty = 0;
{\rbrace}\endCode


\Section
XOR draw cursor at current {\tt {\dollar}crsr{\underline}rect}


\Code
cursor{\underline}toggle({\dollar})
{\lbrace}
        {\dollar}crsr{\underline}on {\caret}= 1;
        XtFillRectangle({\dollar}, {\dollar}gc[CRSR], {\ampersand} {\dollar}crsr{\underline}rect );
{\rbrace}\endCode


\Section
\Code
cursor{\underline}undraw({\dollar})
{\lbrace}
    if( {\dollar}crsr{\underline}on ) cursor{\underline}toggle({\dollar});
{\rbrace}\endCode


\Section
remove old cursor and draw new cursor on window


\Code
cursor{\underline}xdraw({\dollar}, XRectangle * r)
{\lbrace}
    cursor{\underline}undraw({\dollar});
    {\dollar}crsr{\underline}rect = *r;
    cursor{\underline}toggle({\dollar});
{\rbrace}\endCode


\Section
\Code
calculate{\underline}prefered{\underline}size({\dollar}, int * w, int * h)
{\lbrace}
  calculate{\underline}size({\dollar},w,h);
  (*w) += 20;
  (*h) += 20;
{\rbrace}\endCode


\Section
\Code
calculate{\underline}size({\dollar}, int * w, int * h)
{\lbrace}
    *w = {\dollar}txt.width;
    *h = {\dollar}txt.height;
    /*    xft{\underline}calc{\underline}string{\underline}size({\dollar}, {\dollar}xftFont, {\dollar}label, w, h ); */
{\rbrace}\endCode


\Section
\Code
copy{\underline}str({\dollar}, char * s)
{\lbrace}
        m{\underline}clear( {\dollar}str{\underline}m );
        m{\underline}write( {\dollar}str{\underline}m, 0, s, strlen(s) );
        m{\underline}putc( {\dollar}str{\underline}m, '.' );
        str{\underline}to{\underline}glyphlist({\dollar});
        {\dollar}dirty = 1;
{\rbrace}\endCode


\Section
\Code
reinit({\dollar})
{\lbrace}
        free{\underline}colors({\dollar});
        alloc{\underline}colors({\dollar});

        int w,h;
        calculate{\underline}prefered{\underline}size({\dollar},{\ampersand}w,{\ampersand}h);
        make{\underline}resize{\underline}request({\dollar},w,h);
{\rbrace}\endCode


\Section
\Code
make{\underline}resize{\underline}request({\dollar}, int  w, int  h)
{\lbrace}
        Dimension w{\underline}out, h{\underline}out;
        if( {\dollar}width == w {\ampersand}{\ampersand} {\dollar}height == h ) return;
        if( XtMakeResizeRequest({\dollar},w,h, {\ampersand}w{\underline}out, {\ampersand}h{\underline}out ) ==
            XtGeometryAlmost ) XtMakeResizeRequest({\dollar},w{\underline}out,h{\underline}out,NULL,NULL );
{\rbrace}\endCode


\Section
\Code
alloc{\underline}colors({\dollar})
{\lbrace}

  XGCValues     values = {\lbrace}
                foreground: {\dollar}background{\underline}pixel,
                graphics{\underline}exposures: False,
                line{\underline}width: {\dollar}lineWidth {\rbrace};
  uint          mask = GCForeground {\bar} GCGraphicsExposures {\bar}  GCLineWidth;
  {\dollar}gc[BG] = XtGetGC({\dollar}, mask, {\ampersand}values);
  values.foreground = {\dollar}foreground;
  {\dollar}gc[FG] = XtGetGC({\dollar},mask,{\ampersand}values);

  mask = GCForeground {\bar} GCBackground {\bar} GCFunction;
  values.foreground = {\dollar}cursorColor {\caret} {\dollar}background{\underline}pixel;
  values.background = {\dollar}background{\underline}pixel;
  values.function = GXxor;
  {\dollar}gc[CRSR] = XtGetGC({\dollar}, mask, {\ampersand}values);

  xft{\underline}color{\underline}alloc( {\dollar}, {\dollar}background{\underline}pixel, {\dollar}xcol+BG );
  xft{\underline}color{\underline}alloc( {\dollar}, {\dollar}foreground,       {\dollar}xcol+FG );
  xft{\underline}color{\underline}alloc( {\dollar}, {\dollar}cursorColor,      {\dollar}xcol+CRSR );
{\rbrace}\endCode


\Section
\Code
free{\underline}colors({\dollar})
{\lbrace}
        int i;
        Display *dpy = XtDisplay({\dollar});
        for(i=0;i{\langle}2;i++) {\lbrace}
                XtReleaseGC({\dollar},{\dollar}gc[i]);
                XftColorFree(dpy, DefaultVisual(dpy, DefaultScreen(dpy)),
                 None, {\dollar}xcol+i);
        {\rbrace}
{\rbrace}\endCode


\Section
draw glyphs from index {\tt start} to index {\tt stop}


\Code
draw{\underline}glyphs{\underline}spec({\dollar}, int  start, int  stop)
{\lbrace}
    int i, xa = 0;
    if( start {\langle} 0 ) start = 0;
    if( stop {\rangle} m{\underline}len({\dollar}glyph{\underline}m) ) stop = m{\underline}len({\dollar}glyph{\underline}m)-1;
    if( stop {\langle} start ) return;

    for( i=start; i {\langle}= stop; i++ ) {\lbrace}
        get{\underline}glyph{\underline}spec({\dollar},i)-{\rangle}x = xa;
        xa += INT({\dollar}glyph{\underline}xoff{\underline}m,i);
    {\rbrace}

    XftDrawGlyphFontSpec( {\dollar}draw, {\dollar}xcol + FG,
                          mls({\dollar}glyph{\underline}m, start), stop-start +1 );
{\rbrace}\endCode


\Section
draw glyphs from $left_char_offset until width exceeded or end of string


\Code
draw{\underline}glyphs({\dollar})
{\lbrace}
    if( {\dollar}dirty{\underline}str ) str{\underline}to{\underline}glyphlist({\dollar});

    int nglyphs = m{\underline}len({\dollar}glyph{\underline}m) -1;

    int width = {\dollar}pix.r.width;
    int p = {\dollar}left{\underline}char{\underline}offset;
    while( p {\langle} nglyphs ) {\lbrace}
        width -= INT({\dollar}glyph{\underline}xoff{\underline}m,p);
        if( width {\langle} 0 ) break;
        p++;
    {\rbrace}

    draw{\underline}glyphs{\underline}spec({\dollar}, {\dollar}left{\underline}char{\underline}offset, p-1 );
{\rbrace}\endCode


\Section
\Code
unicode{\underline}to{\underline}glyphspec({\dollar}, uint  unicode, XftGlyphFontSpec * specs, int * xp, int * yp, int * xOff)
{\lbrace}
    Display *dpy = XtDisplay({\dollar});
    static FT{\underline}UInt glyph{\underline}fallback = 0;
    FT{\underline}UInt glyphindex;
    XGlyphInfo extents;
    XftFont *fnt = {\dollar}xftFont;

    if( !glyph{\underline}fallback )
        glyph{\underline}fallback = XftCharIndex(dpy,fnt, '{\hashmark}' );

    fnt = font{\underline}unicode{\underline}find({\dollar}fontName, unicode,(int*) {\ampersand}glyphindex );
    if(!glyphindex) {\lbrace}
             fnt={\dollar}xftFont;
             glyphindex = glyph{\underline}fallback;
    {\rbrace}
    XftGlyphExtents( dpy, fnt, {\ampersand}glyphindex, 1, {\ampersand}extents );
    specs-{\rangle}font = fnt;
    specs-{\rangle}glyph = glyphindex;
    specs-{\rangle}x = *xp;
    specs-{\rangle}y = *yp;
    *xp += extents.xOff;
    *yp += extents.yOff;
    *xOff = extents.xOff;
{\rbrace}\endCode


\Section
text aus str_m holen und die listen XftGlyphFontSpec {\tt glyph{\underline}m}
  und int {\tt glyph{\underline}xoff{\underline}m} aufbauen



\Code
str{\underline}to{\underline}glyphlist({\dollar})
{\lbrace}
    int unicode;
    int p;
    int xp,yp;
    yp = {\dollar}xftFont-{\rangle}ascent;
    xp = 0;

    m{\underline}clear( {\dollar}glyph{\underline}m );
    m{\underline}clear( {\dollar}glyph{\underline}xoff{\underline}m );
    p=0;
    while( (unicode = m{\underline}utf8char({\dollar}str{\underline}m, {\ampersand}p)) != -1 )
        {\lbrace}
            unicode{\underline}to{\underline}glyphspec( {\dollar}, unicode, m{\underline}add({\dollar}glyph{\underline}m), {\ampersand}xp, {\ampersand}yp,
                                  m{\underline}add({\dollar}glyph{\underline}xoff{\underline}m) );
        {\rbrace}

    {\dollar}txt.x = 0;
    {\dollar}txt.y = 0;
    {\dollar}txt.width  = xp;
    {\dollar}txt.height = {\dollar}xftFont-{\rangle}ascent + {\dollar}xftFont-{\rangle}descent+2;

    {\dollar}dirty = 1;         /* ggf. update der bildschirmausgabe */
    {\dollar}dirty{\underline}str = 0;     /* glyphlist ist aktualisiert */
{\rbrace}\endCode


\Section
\Code
int  char{\underline}offset({\dollar}, int  index)
{\lbrace}
    int i,p1=0;
    for(i=0;i{\langle}index;i++)
        m{\underline}utf8char( {\dollar}str{\underline}m, {\ampersand}p1 );
    return p1;
{\rbrace}\endCode


\Section
einfügen des übergebenen zeichencodes in den String
  und neuberechnung der glyphs


\Code
insert{\underline}char{\underline}at{\underline}cursor({\dollar}, char * buf, int  len)
{\lbrace}
    int p = char{\underline}offset({\dollar},{\dollar}crsr{\underline}pos);
    m{\underline}ins( {\dollar}str{\underline}m, p, len );
    m{\underline}write( {\dollar}str{\underline}m, p, buf, len );
    str{\underline}to{\underline}glyphlist({\dollar});
    {\dollar}dirty = 1;
{\rbrace}\endCode


\Section
\Code
remove{\underline}char{\underline}at{\underline}cursor({\dollar})
{\lbrace}
    int p1=0, p2=0;
    if( {\dollar}crsr{\underline}pos {\rangle}= m{\underline}len({\dollar}glyph{\underline}m)-1 ) return;
    m{\underline}del( {\dollar}glyph{\underline}m, {\dollar}crsr{\underline}pos );
    m{\underline}del( {\dollar}glyph{\underline}xoff{\underline}m, {\dollar}crsr{\underline}pos );
    p1 = char{\underline}offset({\dollar},{\dollar}crsr{\underline}pos);
    p2=p1; m{\underline}utf8char( {\dollar}str{\underline}m, {\ampersand}p2 );
    m{\underline}remove( {\dollar}str{\underline}m, p1, p2-p1 );
    {\dollar}dirty     = 1;
{\rbrace}\endCode


\Section
\Code
char  scroll{\underline}left({\dollar})
{\lbrace}
    if( {\dollar}left{\underline}char{\underline}offset {\langle} m{\underline}len({\dollar}glyph{\underline}m)-1 ) {\lbrace}
        {\dollar}left{\underline}char{\underline}offset++;
        {\dollar}dirty = 1;
        return 0;
    {\rbrace}
    return -1;
{\rbrace}\endCode


\Section
\Code
char  scroll{\underline}right({\dollar})
{\lbrace}
    if( {\dollar}left{\underline}char{\underline}offset {\langle}= 0 ) return -1;
    {\dollar}left{\underline}char{\underline}offset--;
    {\dollar}dirty = 1;
    return 0;
{\rbrace}\endCode


\Section
coordinate transformation functions
  there are three coordinate systems:
  Character Index inside Text
  Character Pixel Position inside visible area
  Character Pixel Position inside text
  This function returns the Position of the left offset inside the whole text area



\Code
int  px{\underline}left{\underline}offset({\dollar})
{\lbrace}
    return px{\underline}char({\dollar},{\dollar}left{\underline}char{\underline}offset);
{\rbrace}\endCode


\Section
this function returns the poosition of the character at {\tt index} inside the whole text area
  if you subtract the px_left_offset() you get the Position inside the visibile area



\Code
int  px{\underline}char({\dollar}, int  index)
{\lbrace}
    int x=0,i;
    for(i=0;i{\langle}index;i++)
        x+= INT({\dollar}glyph{\underline}xoff{\underline}m, i);
    return x;
{\rbrace}\endCode


\Section
this function returns the pixel postion relative to the visible area



\Code
int  screen{\underline}px({\dollar}, int  index)
{\lbrace}
    return px{\underline}char({\dollar},index) - px{\underline}left{\underline}offset({\dollar});
{\rbrace}\endCode


\Section
 if cursor is not visible after moving,
   scroll display to the left
   to scroll set {\tt left{\underline}char{\underline}offset}



\Code
cursor{\underline}right({\dollar})
{\lbrace}
    int xa = 0;
    if( {\dollar}crsr{\underline}pos {\rangle}= m{\underline}len( {\dollar}glyph{\underline}m ) -1 ) return;

    {\dollar}crsr{\underline}pos++;
    /* das zeichen unter dem cursor sollte komplett sichtbar sein */
    while(1) {\lbrace}
        xa = screen{\underline}px({\dollar}, {\dollar}crsr{\underline}pos );
        xa += INT({\dollar}glyph{\underline}xoff{\underline}m,{\dollar}crsr{\underline}pos);
        if( xa {\langle}= {\dollar}pix.r.width ) break;
        if( scroll{\underline}left({\dollar}) ) break;
    {\rbrace}
{\rbrace}\endCode


\Section
 if cursor is not visible after moving,
   scroll display to the right



\Code
cursor{\underline}left({\dollar})
{\lbrace}
    int xa;
    if( {\dollar}crsr{\underline}pos {\langle}= 0 ) return;

    {\dollar}crsr{\underline}pos--;
    while(1) {\lbrace}
      xa = screen{\underline}px({\dollar}, {\dollar}crsr{\underline}pos );
      if( xa {\rangle}= 0 ) break;
      if( scroll{\underline}right({\dollar}) ) break;
    {\rbrace}
{\rbrace}\endCode


\Section
callback function, called by XtGetSelectionValue()


\Code
RequestSelection({\dollar}, XtPointer  client, Atom * selection, Atom * type, XtPointer  value, unsigned  long * length, int * format)
{\lbrace}
    char *s = value;
    int len = *length;
    if( is{\underline}empty(s) {\bar}{\bar} len {\langle}= 0 ) return;
    insert{\underline}char{\underline}at{\underline}cursor({\dollar},s,len);
    while( utf8char({\ampersand}s) != -1 ) cursor{\underline}right({\dollar});
    paint({\dollar});
{\rbrace}\endCode


\End\bye
